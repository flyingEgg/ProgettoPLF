\documentclass[a4paper,11pt]{article}
\usepackage[utf8]{inputenc}
\usepackage[italian]{babel}
\usepackage{listings}
\usepackage{color}
\usepackage{hyperref}
\usepackage{amsmath}
\usepackage{amssymb}
\usepackage{geometry}
\geometry{a4paper, margin=1in}

% Configurazione del pacchetto listings per il codice
\definecolor{codegreen}{rgb}{0,0.6,0}
\definecolor{codegray}{rgb}{0.5,0.5,0.5}
\definecolor{codepurple}{rgb}{0.58,0,0.82}
\definecolor{backcolour}{rgb}{0.95,0.95,0.92}

\lstdefinestyle{mystyle}{
    backgroundcolor=\color{backcolour},
    commentstyle=\color{codegreen},
    keywordstyle=\color{blue},
    numberstyle=\tiny\color{codegray},
    stringstyle=\color{codepurple},
    basicstyle=\ttfamily\footnotesize,
    breakatwhitespace=false,
    breaklines=true,
    captionpos=b,
    keepspaces=true,
    numbers=left,
    numbersep=5pt,
    showspaces=false,
    showstringspaces=false,
    showtabs=false,
    tabsize=2
}

\lstset{style=mystyle}

\title{Relazione sul progetto di raccomandazione di canzoni}
\author{Giaconi Christian, Giacomo Rossi \\\texttt{Matricola: 314045, 314671}}
\date{\today}

\begin{document}

\maketitle

\section{Specifica del problema}
Il progetto consiste nello sviluppo di un sistema avanzato di raccomandazione di canzoni. Sono state create due implementazioni: una in Haskell e una in Prolog, rispettando i seguenti requisiti:
\begin{itemize}
    \item L'utente fornisce un file di testo contenente un elenco di canzoni nel formato:
    \begin{quote}
        \texttt{Titolo,Artista,Genere,Punteggio}
    \end{quote}
    dove \texttt{Punteggio} è un valore intero tra 1 e 10.
    \item L'utente specifica i generi musicali preferiti e un peso associato per calcolare il punteggio ponderato delle canzoni.
    \item Le canzoni vengono ordinate in base al punteggio ponderato e presentate come una classifica numerata.
\end{itemize}

\section{Analisi del problema}
\subsection{Input}
\begin{itemize}
    \item Un file di testo contenente canzoni con titolo, artista, genere e punteggio.
    \item Una lista di generi musicali preferiti.
    \item Un peso per i generi preferiti.
\end{itemize}

\subsection{Output}
\begin{itemize}
    \item Una classifica di canzoni ordinate per punteggio ponderato.
    \item Dettagli per ogni canzone nella classifica, inclusi titolo, artista, genere e punteggio ponderato.
\end{itemize}

\section{Progettazione dell'algoritmo}
\subsection{Haskell}
L'implementazione in Haskell è suddivisa in una sezione funzionale e una non funzionale:
\begin{itemize}
    \item \textbf{Sezione funzionale:} Contiene funzioni pure per la gestione e l'elaborazione dei dati (es. parsing del file, calcolo del punteggio ponderato, ordinamento delle canzoni).
    \item \textbf{Sezione non funzionale:} Gestisce l'interazione con l'utente e le operazioni di input/output.
\end{itemize}

\subsection{Prolog}
L'implementazione in Prolog utilizza fatti per rappresentare le canzoni e regole per calcolare il punteggio ponderato e ordinare le canzoni. Le funzionalità principali includono:
\begin{itemize}
    \item Caricamento dinamico delle canzoni da file.
    \item Definizione e modifica dei generi preferiti con pesi associati.
    \item Calcolo della classifica basata sui punteggi ponderati.
\end{itemize}

\section{Implementazione dell'algoritmo}
\subsection{Codice Haskell}
\begin{lstlisting}[language=Haskell, caption=Estratto del codice Haskell]
-- Funzione per calcolare il punteggio ponderato
arricchisci :: [String] -> Double -> [Canzone] -> [(Double, Canzone)]
arricchisci _ _ [] = []
arricchisci generiPreferiti peso (c:cs) =
    let genereMinuscolo = map toLower (genere c)
        punteggioPonderato = if genereMinuscolo `elem` generiPreferiti
                             then fromIntegral (punteggio c) * peso
                             else fromIntegral (punteggio c)
    in (punteggioPonderato, c) : arricchisci generiPreferiti peso cs

-- Funzione per stampare la classifica
stampaClassifica :: [(Double, Canzone)] -> IO ()
stampaClassifica raccomandate =
    mapM_ stampaConPosizione (zip [1..] raccomandate)
  where
    stampaConPosizione (pos, (punteggioPonderato, Canzone titolo artista genere _)) = do
        putStrLn $ "#" ++ show pos ++ " - " ++ titolo
        putStrLn $ "   Artista: " ++ artista
        putStrLn $ "   Genere: " ++ genere
        putStrLn $ "   Punteggio ponderato: " ++ show punteggioPonderato
\end{lstlisting}

\subsection{Codice Prolog}
\begin{lstlisting}[language=Prolog, caption=Estratto del codice Prolog]
% Calcola il punteggio ponderato di una canzone
punteggio_ponderato(Titolo, PunteggioPonderato) :-
    canzone(Titolo, _, Genere, Punteggio),
    peso_genere(Genere, Peso),
    PunteggioPonderato is Punteggio * Peso.

% Ottiene la classifica delle canzoni ordinata per punteggio ponderato
classifica_ordinata(Ordinata) :-
    findall(Punteggio-Titolo, punteggio_ponderato(Titolo, Punteggio), Punteggi),
    sort(1, @>=, Punteggi, Ordinata).

% Stampa la classifica delle canzoni
stampa_classifica :- 
    classifica_ordinata(Ordinata),
    stampa_canzoni(Ordinata, 1).
\end{lstlisting}

\section{Testing}
\subsection{Haskell}
\begin{itemize}
    \item \textbf{Input:} File \texttt{canzoni.txt} contenente canzoni e generi preferiti \texttt{Reggaeton, Salsa} con peso \texttt{1.5}.
    \item \textbf{Output atteso:} Classifica numerata con canzoni di Reggaeton e Salsa in cima.
\end{itemize}

\subsection{Prolog}
\begin{itemize}
    \item \textbf{Input:} Fatti iniziali con generi preferiti definiti dinamicamente.
    \item \textbf{Output atteso:} Classifica ordinata e corretta in base ai pesi e punteggi.
\end{itemize}

\section{Conclusione}
Il progetto ha dimostrato l'uso di Haskell e Prolog per implementare sistemi di raccomandazione basati su logiche differenti. Entrambe le implementazioni soddisfano i requisiti e consentono estensioni future, come l'aggiunta di funzionalità o il miglioramento delle interfacce utente.

\end{document}
