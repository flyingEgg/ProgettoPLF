\documentclass[a4paper,11pt]{article}
\usepackage[utf8]{inputenc}
\usepackage[italian]{babel}
\usepackage{amsmath, amssymb}
\usepackage{listings}
\usepackage{color}
\usepackage{hyperref}
\usepackage{geometry}
\geometry{a4paper, margin=1in}

% Colors for code listings
\definecolor{commentgray}{gray}{0.5}
\definecolor{stringgreen}{rgb}{0,0.6,0}
\definecolor{keywordblue}{rgb}{0,0,0.6}

% Listing settings for code
\lstset{
    language=Haskell,
    basicstyle=\ttfamily\small,
    keywordstyle=\color{keywordblue},
    stringstyle=\color{stringgreen},
    commentstyle=\color{commentgray}\itshape,
    breaklines=true,
    frame=single,
    numbers=left,
    numberstyle=\tiny,
    tabsize=2,
    showstringspaces=false,
    captionpos=b
}

\title{\textbf{Relazione del Progetto di Raccomandazione di Canzoni}}
\author{Giaconi Christian, Giacomo Rossi\\Matricola: 314045, 314671}
\date{\today}

\begin{document}

\maketitle

\section{Specifica del problema}
Il progetto consiste nell'implementazione di un sistema di raccomandazione musicale in due linguaggi: Haskell e Prolog. L'obiettivo del programma è di suggerire una classifica di canzoni all'utente in base alle sue preferenze di genere musicale, utilizzando un sistema di punteggio ponderato per priorizzare i risultati. Il contesto applicativo prevede l'uso del programma per scopi personali o di analisi musicale, caricando dati da file strutturati.

Le funzionalità principali includono:
\begin{itemize}
    \item Caricamento di un elenco di canzoni da un file di testo.
    \item Specifica dei generi preferiti e di un peso numerico per valorizzarli.
    \item Calcolo dei punteggi ponderati delle canzoni.
    \item Generazione di una classifica ordinata in base ai punteggi calcolati.
\end{itemize}

\section{Analisi del problema}

\subsection{Dati in ingresso}
Il programma richiede:
\begin{itemize}
    \item Un file contenente un elenco di canzoni, con ciascuna riga nel formato:
    \begin{verbatim}
    Titolo,Artista,Genere,Punteggio
    \end{verbatim}
    \item Una lista di generi musicali preferiti specificata dall'utente.
    \item Un valore numerico (peso) per i generi preferiti, che amplifica la rilevanza delle canzoni appartenenti a quei generi.
\end{itemize}

\subsection{Dati in uscita}
Il programma produce:
\begin{itemize}
    \item Una lista di canzoni ordinata in base ai punteggi ponderati calcolati, stampata come una classifica numerata.
    \item Messaggi esplicativi per guidare l'utente nelle interazioni.
\end{itemize}

\subsection{Relazioni tra i dati}
\begin{itemize}
    \item Il punteggio ponderato di ogni canzone è calcolato moltiplicando il suo punteggio originale per il peso associato al suo genere, se presente tra i generi preferiti.
    \item L'ordinamento della classifica avviene in ordine decrescente rispetto ai punteggi ponderati.
\end{itemize}

\section{Progettazione dell'algoritmo}

\subsection{Scelte di progetto}
\begin{itemize}
    \item In Haskell, i dati delle canzoni sono rappresentati tramite un tipo di dato \texttt{Canzone} per migliorare leggibilità e modularità.
    \item In Prolog, le canzoni sono definite come fatti dinamici per consentire la manipolazione dei dati durante l'esecuzione.
\end{itemize}

\subsection{Passi dell'algoritmo}
\begin{enumerate}
    \item Caricamento dei dati da file.
    \item Analisi e parsing delle informazioni delle canzoni.
    \item Inserimento delle preferenze dell'utente.
    \item Calcolo dei punteggi ponderati.
    \item Ordinamento delle canzoni e generazione della classifica.
\end{enumerate}

\section{Implementazione dell'algoritmo}

\subsection{Implementazione in Haskell}
Il programma in Haskell utilizza funzioni pure per elaborare i dati e azioni IO per interagire con l'utente.

Esempio di calcolo dei punteggi ponderati:
\begin{lstlisting}[caption=Calcolo dei punteggi ponderati in Haskell]
arricchisci :: [String] -> Double -> [Canzone] -> [(Double, Canzone)]
arricchisci _ _ [] = []
arricchisci generiPreferiti peso (c:cs) =
    let genereMinuscolo = map toLower (genere c)
        punteggioPonderato = if genereMinuscolo `elem` generiPreferiti
                             then fromIntegral (punteggio c) * peso
                             else fromIntegral (punteggio c)
    in (punteggioPonderato, c) : arricchisci generiPreferiti peso cs
\end{lstlisting}

\subsection{Implementazione in Prolog}
Il programma in Prolog sfrutta predicati per rappresentare i dati e regole logiche per elaborarli.

Esempio di ordinamento delle canzoni:
\begin{lstlisting}[caption=Ordinamento delle canzoni in Prolog]
classifica_ordinata(Ordinata) :-
    findall(Punteggio-Titolo, punteggio_ponderato(Titolo, Punteggio), Punteggi),
    sort(1, @>=, Punteggi, Ordinata).
\end{lstlisting}

\section{Testing}

\subsection{Esempi di input e output}
Input di esempio (file \texttt{canzoni.txt}):
\begin{verbatim}
Despacito,Luis Fonsi,Reggaeton,8
Danza Kuduro,Don Omar,Reggaeton,9
Bachata Rosa,Juan Luis Guerra,Bachata,9
\end{verbatim}

Output per generi preferiti \texttt{Reggaeton} con peso 1.5:
\begin{verbatim}
#1 - Danza Kuduro
   Artista: Don Omar
   Genere: Reggaeton
   Punteggio ponderato: 13.5
#2 - Despacito
   Artista: Luis Fonsi
   Genere: Reggaeton
   Punteggio ponderato: 12.0
\end{verbatim}

\subsection{Verifica dei risultati e limitazioni}
Il programma funziona correttamente per file ben formattati. Estensioni future potrebbero includere:
\begin{itemize}
    \item Validazione dei dati in ingresso.
    \item Supporto per più criteri di raccomandazione (es. anno, durata).
\end{itemize}

\end{document}
