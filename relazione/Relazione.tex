\documentclass[a4paper,11pt]{article}
\usepackage[utf8]{inputenc}
\usepackage[italian]{babel}
\usepackage{amsmath, amssymb}
\usepackage{listings}
\usepackage{color}
\usepackage{xcolor}
\usepackage{times}
\usepackage{geometry}
\usepackage{graphicx}
\usepackage{listings}
\usepackage{hyperref}
\usepackage[utf8]{inputenc}

\geometry{a4paper, margin=1in}

% Colors for code listings
\definecolor{commentgray}{gray}{0.5}
\definecolor{stringgreen}{rgb}{0,0.6,0}
\definecolor{keywordblue}{rgb}{0,0,0.6}

% Listing settings for code
\lstset{
    basicstyle=\ttfamily\small,
    keywordstyle=\color{keywordblue},
    stringstyle=\color{stringgreen},
    commentstyle=\color{commentgray}\itshape,
    breaklines=true,
    frame=none,
    numbers=none,
    numberstyle=\tiny,
    tabsize=2,
    showstringspaces=false,
    captionpos=b
}

\geometry{top=3cm, bottom=3cm, left=3cm, right=3cm}


\begin{document}

% Inserisci il logo (spazio riservato)
\begin{center}
    \includegraphics[width=3cm]{logo.png}
\end{center}

% Titolo dell'università
\begin{center}
    \textsc{\textbf{Università degli Studi di Urbino}}\\
    \textsc{Dipartimento di Scienze Pure e Applicate}\\
    \textsc{Corso di Laurea in Informatica Applicata}
\end{center}

\vspace{2cm}

% Titolo del progetto
\begin{center}
    \textbf{\Large Progetto di Programmazione Logica e Funzionale}
\end{center}

\vspace{2cm}

% Informazioni dello studente
\begin{center}
    \textbf{di Giaconi Christian, Giacomo Rossi} \\ % Spazio per il nome dello studente
    \textbf{Matricola 314045, 314671} \\ % Spazio per la matricola
    \textbf{Anno di corso: terzo}
\end{center}

\vspace{2cm}

% Anno accademico e docente
\begin{center}
    \textbf{Anno Accademico 2023/2024 - Sessione invernale}\\
    \vspace{1cm}
    \textbf{Docente: Prof. Marco Bernardo}
\end{center}

\newpage
\tableofcontents

\newpage
\section{Specifica del problema}
\textit{Scrivere un programma Haskell e un programma Prolog per implementare un sistema di raccomandazione di canzoni. Il programma legge le canzoni espresse in quattro attributi: titolo, artista, genere musicale e punteggio; da un file con valori separati da virgole, il cui nome viene acquisito da tastiera e le suggerisce all’utente basandosi sulle sue preferenze musicali, utilizzando un punteggio di gradimento acquisito da tastiera e configurato su uno o più generi, per creare una classifica con le canzoni ordinate secondo il punteggio ponderato da quello di gradimento del o dei generi musicali.}

\newpage
\section{Analisi del problema}
\subsection{Dati in ingresso}
\begin{itemize}
    \item Un file di valori separati da virgole contenenti le canzoni nel formato:
    \begin{verbatim}
    Titolo,Artista,Genere,Punteggio
    \end{verbatim}
    \item Una lista di generi preferiti.
    \item Pesi numerici assegnati ai generi preferiti.
\end{itemize}

\subsection{Dati in uscita}
\begin{itemize}
    \item Una classifica ordinata di canzoni basata sui punteggi ponderati.
    \item Una lista dei generi impostati come preferiti con il rispettivo peso associato.
\end{itemize}

\subsection{Relazioni intercorrenti}
Ogni canzone è definita da quattro attributi principali:
\begin{itemize}
    \item \textbf{Titolo}: la denominazione della canzone.  
    \item \textbf{Artista}: l’autore o il gruppo musicale che l’ha prodotta.  
    \item \textbf{Genere}: la categoria musicale di appartenenza (ad esempio, pop, rock, jazz).  
    \item \textbf{Punteggio}: una valutazione numerica che rappresenta il grado di apprezzamento della canzone.  
\end{itemize}

Il \textbf{Punteggio Ponderato} viene calcolato secondo la seguente formula:
\begin{center}
    \textit{
    \[
    \text{Punteggio Ponderato} =
    \begin{cases} 
      \text{Punteggio} \times \text{Peso Genere}, & \text{se il genere è preferito dall’utente} \\
      \text{Punteggio originale}, & \text{altrimenti}
    \end{cases}
    \]
    }
\end{center}

\noindent
Il \textbf{Peso Genere} rappresenta un coefficiente che riflette il livello di preferenza dell’utente verso un determinato genere musicale. Ciò consente di personalizzare la valutazione delle canzoni, attribuendo maggiore rilevanza ai generi di interesse senza tuttavia escludere le altre categorie, bilanciando così criteri oggettivi e preferenze soggettive.  

\newpage
\section{Progettazione dell'algoritmo}
\subsection{Scelte di progetto}
Per sviluppare il sistema di raccomandazione musicale, abbiamo adottato approcci distinti per ciascuno dei linguaggi richiesti, Haskell e Prolog, al fine di sfruttare al meglio le peculiarità di ciascuno.

In \textbf{Haskell}, abbiamo deciso di rappresentare le canzoni utilizzando tipi di dati strutturati. Questa scelta ci consente di garantire una gestione chiara e sicura delle informazioni musicali, sfruttando le capacità del sistema di tipi statico per ridurre errori e facilitare l'implementazione delle operazioni. Le canzoni vengono caricate da un file di testo, per consentire l'integrazione di dataset esterni senza modificare il codice sorgente. La scelta di utilizzare un file di testo è stata motivata dalla necessità di mantenere il sistema flessibile e facilmente adattabile a dataset di dimensioni variabili.

In \textbf{Prolog}, invece, abbiamo optato per una rappresentazione tramite predicati dinamici. Le canzoni sono caricate all’interno del programma attraverso il predicato \texttt{carica\_canzoni/0}, che utilizza \texttt{assertz/1} per definire dinamicamente ogni canzone come un fatto nella base di conoscenza. Questo approccio ci permette di manipolare i dati con la semantica dichiarativa propria di Prolog, evitando problematiche legate all’importazione di file esterni. Ogni canzone è rappresentata da un predicato \texttt{canzone/4}, che descrive il titolo, l’artista, il genere e il punteggio. La struttura dati utilizzata è completamente dinamica, consentendo aggiunte o modifiche in tempo reale senza la necessità di ridefinire l’intero dataset.

\begin{itemize}
    \item \textbf{Haskell:}
    \begin{itemize}
        \item L’uso di file di testo consente di estendere il dataset semplicemente aggiungendo nuove righe al file.
        \item Le funzioni per la manipolazione delle stringhe e la costruzione di tipi di dati strutturati in Haskell rendono questa operazione relativamente semplice e modulare.
        \item Inoltre, Haskell permette un trattamento funzionale dei dati, come la mappatura e il filtraggio, che è particolarmente utile per calcolare e ordinare i punteggi ponderati.
    \end{itemize}
    \item \textbf{Prolog:}
    \begin{itemize}
        \item La rappresentazione tramite predicati è particolarmente adatta per le relazioni tra canzoni, generi e preferenze dell’utente.
        \item Caricare le canzoni come fatti dinamici permette di utilizzare direttamente la logica del linguaggio per effettuare operazioni come il filtraggio e l’ordinamento basato sui punteggi.
        \item Questo approccio evita la necessità di definire una sintassi di input complessa, sfruttando la semplicità dichiarativa propria di Prolog.
    \end{itemize}
    \item \textbf{Gestione dell’input e output:}
    \begin{itemize}
        \item In Haskell, il file di testo richiede un parsing iniziale per costruire una lista di canzoni.
        \item In Prolog, invece, la definizione diretta nel predicato \texttt{carica\_canzoni/0} semplifica l’esecuzione e permette di focalizzarsi sull’elaborazione delle raccomandazioni piuttosto che sulla gestione dell’input.
    \end{itemize}
\end{itemize}

\newpage
\subsection{Passi dell'algoritmo}
I passi principali dell'algoritmo, comuni sia per l'implementazione in Haskell che in Prolog, sono i seguenti:

\begin{enumerate}
    \item \textbf{Caricare le canzoni:} 
    \begin{itemize}
        \item In Haskell, le canzoni vengono caricate da un file di testo, che permette la costruzione di una lista di canzoni.
        \item In Prolog, le canzoni sono definite all’interno del programma attraverso il predicato \texttt{carica\_canzoni/0}.
    \end{itemize}
    \item \textbf{Inserire le preferenze dell'utente:} 
    L’utente fornisce i generi musicali preferiti e i relativi pesi, che influenzeranno il calcolo dei punteggi ponderati.
    \item \textbf{Calcolare i punteggi ponderati:} 
    Per ogni canzone, viene calcolato un punteggio combinando il peso associato al genere con il punteggio individuale della canzone.
    \item \textbf{Ordinare le canzoni per punteggio ponderato:} 
    Le canzoni vengono ordinate in ordine decrescente rispetto al punteggio ponderato, così da ottenere una classifica delle raccomandazioni.
    \item \textbf{Stampare la classifica:} 
    L’algoritmo produce in output la lista ordinata delle canzoni con i rispettivi punteggi, offrendo una chiara visualizzazione per l’utente.
\end{enumerate}

La ricorsività è utilizzata in entrambi i linguaggi per garantire la gestione dinamica dei dati:

\begin{itemize}
    \item \textbf{Haskell:} 
    Le funzioni ricorsive vengono impiegate per elaborare le liste di canzoni. Ad esempio, il calcolo dei punteggi ponderati viene implementato tramite una funzione che scorre ricorsivamente l’intera lista, applicando il peso corrispondente al genere di ciascuna canzone.
    \item \textbf{Prolog:} 
    L’ordinamento delle canzoni si basa su un predicato ricorsivo che inserisce ogni elemento in una lista ordinata. La dinamica di Prolog consente inoltre di aggiungere o aggiornare i generi preferiti senza ridefinire l’intera base di conoscenza.
\end{itemize}

\newpage
\section{Implementazione dell'algoritmo}
\subsection{Implementazione in Haskell}
Il file \texttt{raccomandazioni.hs} implementa l'algoritmo in Haskell. Un esempio di calcolo dei punteggi ponderati:
\begin{lstlisting}[language=Haskell]
    -- #########################################################
    -- # Corso di Programmazione Logica e Funzionale           #
    -- # Progetto di raccomandazione di canzoni                #
    -- # Studente: Giaconi Christian, Giacomo Rossi            #
    -- # Matricola: 314045, 314671                             #
    -- #########################################################
    
    {- Specifica:
        Scrivere un programma in Haskell per implementare un sistema avanzato di raccomandazione di canzoni.
        Il sistema suggerisce canzoni a un utente in base a:
        - Preferenze per uno o più generi musicali specificati.
        - Un sistema di punteggio ponderato per dare priorità a canzoni più rilevanti.
        L'utente deve fornire un file di testo con le canzoni nel seguente formato:
            Titolo,Artista,Genere,Punteggio
        Dove "Punteggio" è un intero da 1 a 10.
        Le canzoni saranno ordinate in base al punteggio ponderato e filtrate per genere.
    -}
    
    module Main where
    
    import Data.List (sortOn, nub, intercalate)
    import Data.Maybe (mapMaybe)
    import Data.Ord (Down(..))
    import qualified Data.Map as Map
    import System.IO.Error(isDoesNotExistError)
    import Control.Exception (catch, IOException)
    import Text.Read (readMaybe)
    
    -- #########################################################
    -- Definizioni dei tipi di dati
    -- #########################################################
    
    -- | La struttura 'Canzone' rappresenta una canzone con:
    -- - titolo: il titolo della canzone.
    -- - artista: l'artista che la interpreta.
    -- - genere: il genere musicale della canzone.
    -- - punteggio: un punteggio assegnato (da 1 a 10).
    data Canzone = Canzone
        { titolo    :: String
        , artista   :: String
        , genere    :: String
        , punteggio :: Int
        } deriving (Show, Eq)
    
    -- | PesiGeneri è una mappa che associa un genere musicale a un peso
    -- che influenza la priorità delle raccomandazioni.
    type PesiGeneri = Map.Map String Double
    
    -- #########################################################
    -- Main: Menu interattivo
    -- #########################################################
    
    -- | Funzione principale che avvia il menu interattivo.
    main :: IO ()
    main = menuLoop Nothing Map.empty
    
    -- | Gestisce il menu principale, mantenendo lo stato del sistema:
    -- - maybeCanzoni: un elenco opzionale delle canzoni caricate.
    -- - pesi: i pesi dei generi preferiti, gestiti dall'utente.
    menuLoop :: Maybe [Canzone] -> PesiGeneri -> IO ()
    menuLoop maybeCanzoni pesi = do
        putStrLn "\n--- Sistema di Raccomandazione Musicale ---"
        putStrLn "1. Carica un file con le canzoni"
        putStrLn "2. Gestisci i generi preferiti (aggiungi o modifica)"
        putStrLn "3. Stampa la classifica delle canzoni"
        putStrLn "4. Stampa i generi preferiti con il relativo punteggio"
        putStrLn "5. Esci"
        putStrLn "Seleziona un'opzione:"
        scelta <- getLine
        case scelta of
            "1" -> caricaCanzoni >>= (`menuLoop` pesi) . Just
            "2" -> selezionaGeneriPreferitiEImpostaPesi maybeCanzoni pesi >>= menuLoop maybeCanzoni
            "3" -> raccomandaCanzoni maybeCanzoni pesi >> menuLoop maybeCanzoni pesi
            "4" -> visualizzaGeneriPreferiti pesi >> menuLoop maybeCanzoni pesi
            "5" -> putStrLn "Grazie per aver usato il sistema di raccomandazione. Arrivederci!"
            _   -> putStrLn "Opzione non valida. Riprova." >> menuLoop maybeCanzoni pesi
    
    -- #########################################################
    -- Funzioni di caricamento e gestione dei dati
    -- #########################################################
    
    -- | Carica un file di testo, legge i dati delle canzoni e li
    -- trasforma in una lista di Canzone.
    -- Il file deve avere un formato valido: Titolo,Artista,Genere,Punteggio.
    caricaCanzoni :: IO [Canzone]
    caricaCanzoni = do
        nomeFile <- chiediNomeFile
        contenuto <- readFile nomeFile
        let canzoni = mapMaybe analizzaCanzone (lines contenuto)
        if null canzoni
            then putStrLn "Errore: il file non contiene dati validi! Riprova." >> caricaCanzoni
            else putStrLn "File caricato con successo!" >> return canzoni
    
    -- | Richiede all'utente di inserire il nome del file con le canzoni
    -- e ne effettua una validazione dell'input tramite la funzione validaFile.
    chiediNomeFile :: IO FilePath
    chiediNomeFile = do
        putStrLn "Inserire il nome del file:"
        nomeFile <- getLine
        esito_lettura <- validaFile nomeFile
        case esito_lettura of
            Right () -> return nomeFile  -- Restituisce il nome del file se valido
            Left err -> do
                putStrLn $ "Errore: " ++ err
                chiediNomeFile
    
    -- | Controlla se il nome del file è espresso
    -- correttamente e se tale file esiste.
    validaFile :: FilePath -> IO (Either String ())
    validaFile nomeFile =
        catch (readFile nomeFile >> return (Right ()))
              (\e -> if isDoesNotExistError e
                     then return $ Left "File non trovato!"
                     else return $ Left "Errore durante l'apertura del file.")
    
    -- | Permette all'utente di scegliere
    -- i generi preferiti e assegnare un peso a ciascuno di essi.
    selezionaGeneriPreferitiEImpostaPesi :: Maybe [Canzone] -> PesiGeneri -> IO PesiGeneri
    selezionaGeneriPreferitiEImpostaPesi Nothing pesi = do
        putStrLn "Errore: nessun file caricato. Carica un file prima di continuare."
        return pesi
    selezionaGeneriPreferitiEImpostaPesi (Just canzoni) pesi = do
        let generiDisponibili = nub $ map genere canzoni
        putStrLn $ "Generi disponibili: " ++ intercalate ", " generiDisponibili
        generiSelezionati <- raccogliGeneri generiDisponibili
        aggiornaPesi generiSelezionati pesi
    
    -- | Consente all'utente di inserire i generi
    -- preferiti uno alla volta, terminando con "fine".
    raccogliGeneri :: [String] -> IO [String]
    raccogliGeneri generiDisponibili = do
        putStrLn "Inserisci i generi preferiti uno alla volta. Scrivi 'fine' per terminare."
        loop []
      where
        loop acc = do
            putStrLn "Inserisci un genere preferito:"
            input <- getLine
            if input == "fine"
                then return (nub acc)
                else if input `elem` generiDisponibili
                     then putStrLn ("Genere '" ++ input ++ "' aggiunto ai preferiti.") >> loop (input : acc)
                     else putStrLn "Genere non valido. Riprova." >> loop acc
    
    -- | Consente all'utente di modificare i pesi dei generi preferiti.
    -- Se il genere ha già un peso, l'utente può scegliere di mantenerlo o aggiornarlo.
    aggiornaPesi :: [String] -> PesiGeneri -> IO PesiGeneri
    aggiornaPesi [] pesi = return pesi
    aggiornaPesi (g:gs) pesi = do
        let pesoCorrente = Map.findWithDefault 1.0 g pesi
        putStrLn $ "Peso corrente per il genere '" ++ g ++ "': " ++ show pesoCorrente
        putStrLn "Vuoi aggiornare il peso? (s/n)"
        risposta <- getLine
        if risposta == "s"
            then do
                putStrLn $ "Inserisci il nuovo peso per il genere '" ++ g ++ "':"
                nuovoPeso <- leggiPesoValido
                aggiornaPesi gs (Map.insert g nuovoPeso pesi)
            else do
                putStrLn $ "Peso per il genere '" ++ g ++ "' invariato."
                aggiornaPesi gs pesi
    
    -- #########################################################
    -- Raccomandazioni
    -- #########################################################
    
    -- | Genera e stampa una lista di canzoni consigliate
    -- basandosi sui pesi dei generi e sui punteggi delle canzoni.
    raccomandaCanzoni :: Maybe [Canzone] -> PesiGeneri -> IO ()
    raccomandaCanzoni Nothing _ = putStrLn "Errore: nessun file caricato. Carica un file prima di continuare."
    raccomandaCanzoni (Just canzoni) pesi = do
        let raccomandate = raccomanda pesi canzoni
        if null raccomandate
            then putStrLn "Nessuna canzone trovata con i pesi attuali."
            else stampaClassifica raccomandate
    
    -- #########################################################
    -- Funzioni ausiliarie
    -- #########################################################
    
    -- | Converte una riga di testo in un oggetto Canzone.
    -- Restituisce Nothing se la riga non è formattata correttamente.
    analizzaCanzone :: String -> Maybe Canzone
    analizzaCanzone riga =
        case separaTaglia ',' riga of
            [titolo, artista, genere, punteggioStr]
                | "" `notElem` [titolo, artista, genere, punteggioStr]  -- Controlla che tutte le parti siano non vuote
                , Just punteggio <- readMaybe punteggioStr  -- Prova a leggere il punteggio
                , punteggio >= 1 && punteggio <= 10 -> Just (Canzone titolo artista genere punteggio)  -- Verifica che il punteggio sia valido
            _ -> Nothing  -- Restituisce Nothing se la riga non è valida
    
    -- | Divide una stringa in una lista di stringhe, usando un delimitatore.
    separa :: Char -> String -> [String]
    separa _ "" = []
    separa delimiter string =
        let (primo, resto) = break (== delimiter) string
        in primo : case resto of
            [] -> []
            x -> separa delimiter (dropWhile (== delimiter) (tail x))
    
    -- | Divide una stringa in campi separati, pulendo gli spazi.
    separaTaglia :: Char -> String -> [String]
    separaTaglia delimiter string = map (filter (/= ' ')) (separa delimiter string)
    
    -- | Legge un valore di peso valido inserito dall'utente.
    leggiPesoValido :: IO Double
    leggiPesoValido = do
        input <- getLine
        case readMaybe input of
            Just peso | peso > 0 -> return peso
            _ -> putStrLn "Peso non valido. Riprova." >> leggiPesoValido
    
    -- | Calcola il punteggio ponderato per ogni canzone e le ordina.
    raccomanda :: PesiGeneri -> [Canzone] -> [(Double, Canzone)]
    raccomanda pesi canzoni =
        let arricchite = arricchisci pesi canzoni
        in sortOn (Down . fst) arricchite
    
    -- | Calcola il punteggio ponderato per ogni canzone.
    arricchisci :: PesiGeneri -> [Canzone] -> [(Double, Canzone)]
    arricchisci pesi canzoni =
        [ (fromIntegral (punteggio c) * Map.findWithDefault 1.0 (genere c) pesi, c) | c <- canzoni ]
    
    -- | Stampa le canzoni ordinate con il loro punteggio ponderato.
    stampaClassifica :: [(Double, Canzone)] -> IO ()
    stampaClassifica raccomandate =
        mapM_ stampaConPosizione (zip [1..] raccomandate)
        where
            stampaConPosizione (pos, (punteggioPonderato, Canzone titolo artista genere _)) = do
                putStrLn $ "#" ++ show pos ++ " - " ++ titolo
                putStrLn $ "   Artista: " ++ artista
                putStrLn $ "   Genere: " ++ genere
                putStrLn $ "   Punteggio ponderato: " ++ show punteggioPonderato
                putStrLn "-------------------------------------------"
    
    -- | Visualizza i generi preferiti e i pesi associati.
    visualizzaGeneriPreferiti :: PesiGeneri -> IO ()
    visualizzaGeneriPreferiti pesi
        | Map.null pesi = putStrLn "Nessun genere ancora definito."
        | otherwise = do
            putStrLn "I tuoi generi preferiti e pesi associati sono:"
            mapM_ stampaGenere (Map.toList pesi)
    
    -- | Stampa il genere, concatenato al peso suo relativo
    stampaGenere :: (String, Double) -> IO ()
    stampaGenere (genere, peso) = putStrLn $ genere ++ ": " ++ show peso
\end{lstlisting}

\newpage
\subsection{Implementazione in Prolog}
Il file \texttt{raccomandazioni.pl} implementa l'algoritmo in Prolog. Esempio di ordinamento delle canzoni:
\begin{lstlisting}[language=Prolog]
/* ######################################################### */
/* # Corso di Programmazione Logica e Funzionale           # */
/* # Progetto di raccomandazione di canzoni                # */
/* # Studente: Giaconi Christian, Giacomo Rossi            # */
/* # Matricola: 314045, 314671                             # */
/* ######################################################### */

/* ================================================
   Predicati dinamici
   ================================================ */

/* Predicato che 'canzone/4' memorizza informazioni relative
   alle canzoni caricate. Ogni canzone è rappresentata 
   dai seguenti argomenti: Titolo, Artista, Genere e Punteggio. */
:- dynamic(canzone/4).

/* Predicato che 'genere_preferito/2' associa un peso preferito
   a ciascun genere musicale. Il primo argomento è il Genere,
   il secondo è il Peso associato a quel genere. */
:- dynamic(genere_preferito/2).

/* ================================================
   Predicati principali
   ================================================ */

/* Predicato che 'main' è il punto di ingresso principale.
   Inizializza il programma e avvia il menu interattivo
   per l'utente. */
main :- 
    nl,
    write('Benvenuto nel sistema di raccomandazione musicale!'),
    carica_canzoni,
    loop_menu.

/* Predicato che 'loop_menu' gestisce la selezione delle azioni
   da parte dell'utente nel menu principale. Ogni opzione del menu
   chiama un predicato specifico per eseguire l'azione corrispondente. */
loop_menu :- 
    nl,
    write('======================================'), nl,
    write('Scegli un\'azione: '), nl,
    write('1. Gestisci i generi preferiti (aggiungi o modifica)'), nl,
    write('2. Stampa la classifica delle canzoni'), nl,
    write('3. Stampa la lista dei generi preferiti'), nl,
    write('4. Esci'), nl,
    write('======================================'), nl,
    write('Inserisci la tua scelta: '), nl,
    read(Scelta),
    (   Scelta = 1 -> gestisci_generi_preferiti
    ;   Scelta = 2 -> stampa_classifica
    ;   Scelta = 3 -> mostra_generi_preferiti
    ;   Scelta = 4 -> write('Arrivederci!\n'), halt
    ;   write('Scelta non valida. Riprova.\n')
    ),
    loop_menu.

carica_canzoni :- 
    assertz(canzone('Despacito', 'Luis Fonsi', 'Reggaeton', 8)),
    assertz(canzone('All Eyez On Me', 'Tupac', 'HipHop', 8)),
    assertz(canzone('Danza Kuduro', 'Don Omar', 'Reggaeton', 9)),
    assertz(canzone('Song 2', 'Blur', 'Alternative/Indie', 6)),
    assertz(canzone('Bachata Rosa', 'Juan Luis Guerra', 'Bachata', 9)),
    assertz(canzone('Notturno op 55 no 1', 'Chopin', 'Classica', 6)),
    assertz(canzone('Free Bird', 'Lynyrd Skynyrd', 'Rock', 8)),
    assertz(canzone('Thunderstruck', 'AC/DC', 'Rock', 7)),
    assertz(canzone('Come As You Are', 'Nirvana', 'Rock', 8)),
    assertz(canzone('La Gota Fria', 'Carlos Vives', 'Vallenato', 7)),
    assertz(canzone('Stronger', 'Kanye West', 'HipHop', 9)),
    assertz(canzone('Californication', 'Red Hot Chili Peppers', 'Alternative/Indie', 6)),
    assertz(canzone('Upper Echelon', 'Travis Scott', 'Trap', 7)),
    assertz(canzone('El Cantante', 'Hector Lavoe', 'Salsa', 9)),
    assertz(canzone('Suavemente', 'Elvis Crespo', 'Merengue', 10)),
    assertz(canzone('La Vaca', 'Los Toros Band', 'Merengue', 9)).

/* ================================================
   Predicati per la gestione dei generi preferiti
   ================================================ */

/* Predicato che permette all'utente di selezionare 
   e gestire i generi musicali preferiti. */
gestisci_generi_preferiti :- 
    mostra_generi_disponibili,
    write('Inserisci i tuoi generi preferiti tra apici, uno per volta. Scrivi "fine" per terminare.\n'),
    chiedi_generi_preferiti([]).

/* Predicato che raccoglie i generi preferiti inseriti
   dall'utente e li aggiunge alla lista di preferiti. */
chiedi_generi_preferiti(GeneriPreferiti) :- 
    write('Inserisci un genere preferito: '),
    read(Genere),
    (   Genere == fine
    ->  chiedi_peso_generi(GeneriPreferiti)
    ;   findall(GenereDisponibile, canzone(_, _, GenereDisponibile, _), GeneriDisponibili), 
        elimina_duplicati(GeneriDisponibili, GeneriUnici),
        (   membro(Genere, GeneriUnici)
        ->  append(GeneriPreferiti, [Genere], NuoviGeneri),
            chiedi_generi_preferiti(NuoviGeneri)
        ;   write('Genere non valido. Ecco i generi disponibili:\n'),
            scrivi_lista(GeneriUnici),
            chiedi_generi_preferiti(GeneriPreferiti) 
        )
    ).


/* Predicato che chiede all'utente di inserire
   un peso per ciascun genere musicale preferito. */
chiedi_peso_generi([]).
chiedi_peso_generi([Genere | Altri]) :- 
    format('Inserisci il peso per il genere ~w: ', [Genere]),
    read(Peso),
    (   number(Peso), Peso > 0
    ->  assertz(genere_preferito(Genere, Peso)),
        chiedi_peso_generi(Altri)
    ;   write('Peso non valido. Riprova.\n'),
        chiedi_peso_generi([Genere | Altri]) ).

/* ================================================
   Predicati per la raccomandazione e la classifica
   ================================================ */

/* Predicato che calcola il punteggio ponderato per ogni canzone 
   in base al suo genere e al suo punteggio originale.
   Poi stampa la classifica ordinata delle canzoni. */
stampa_classifica :- 
    nl, write('Calcolando la classifica...'), nl,
    findall(PunteggioPonderato-Titolo, calcola_punteggio_ponderato(Titolo, PunteggioPonderato), Punteggi),
    (   Punteggi == []
    ->  nl, write('Nessuna canzone trovata con punteggio ponderato.'), nl
    ;   ordina_lista(Punteggi, PunteggiOrdinati),
        nl, write('Ecco la classifica delle canzoni:'), nl,
        stampa_canzoni_ordinate(PunteggiOrdinati, 1)
    ).

/* Predicato che stampa le canzoni ordinate in base al punteggio
   ponderato, elencandole con la posizione, il titolo,
   l'artista e il punteggio ponderato. */
stampa_canzoni_ordinate([], _).
stampa_canzoni_ordinate([PunteggioPonderato-Titolo | Rest], Posizione) :- 
    canzone(Titolo, Artista, Genere, _),
    format('~d# ~w (Artista: ~w, Genere: ~w, Punteggio ponderato: ~2f)\n', 
           [Posizione, Titolo, Artista, Genere, PunteggioPonderato]),
    Posizione1 is Posizione + 1,
    stampa_canzoni_ordinate(Rest, Posizione1).

/* Predicato che calcola il punteggio ponderato
   di una canzone in base al suo genere (e al peso preferito associato)
   moltiplicato per il punteggio originale della canzone. */
calcola_punteggio_ponderato(Titolo, PunteggioPonderato) :- 
    canzone(Titolo, _, Genere, Punteggio),
    peso_genere(Genere, Peso),
    PunteggioPonderato is Punteggio * Peso.

/* ================================================
   Predicati ausiliari
   ================================================ */

/* Predicato che mostra i generi preferiti associati
   con il rispettivo peso. */
mostra_generi_preferiti :- 
    findall(Genere-Peso, genere_preferito(Genere, Peso), Generi),
    (   Generi == []
    ->  write('Non è stato definito alcun genere preferito.\n')
    ;   write('I tuoi generi preferiti e i loro pesi:\n'),
        stampa_generi(Generi)
    ).

/* Predicato che restituisce una lista dei generi musicali 
   presenti nel database delle canzoni, evitando duplicati. */
mostra_generi_disponibili :- 
    findall(Genere, canzone(_, _, Genere, _), Generi),
    elimina_duplicati(Generi, GeneriUnici),
    write('Generi disponibili:\n'),
    scrivi_lista(GeneriUnici).

/* Predicato che elimina duplicati da una lista. */
elimina_duplicati([], []).
elimina_duplicati([H|T], [H|T1]) :- 
    non_membro(H, T),
    elimina_duplicati(T, T1).
elimina_duplicati([H|T], T1) :- 
    membro(H, T),
    elimina_duplicati(T, T1).

/* Predicato che verifica se un elemento è membro della lista. */
membro(X, [X|_]).
membro(X, [_|T]) :- 
    membro(X, T).

/* Predicato che verifica se un elemento NON è membro della lista. */
non_membro(_, []).
non_membro(X, [H|T]) :- 
    X \= H,
    non_membro(X, T).

/* Scrive una lista elemento per elemento */
scrivi_lista([]).
scrivi_lista([H|T]) :- 
    write('- '), write(H), nl,
    scrivi_lista(T).

/* Predicato che stampa la lista dei generi preferiti. */
stampa_generi([]).
stampa_generi([Genere-Peso | Rest]) :- 
    format('~w: ~w\n', [Genere, Peso]),
    stampa_generi(Rest).

/* Predicato che 'peso_genere' restituisce il peso di un genere.
   Se non è specificato, viene utilizzato un peso di 1. */
peso_genere(Genere, Peso) :-
    (   genere_preferito(Genere, Peso) 
    ->  true 
    ;   Peso = 1 ).

/* Predicato che ordina una lista in ordine decrescente. */
ordina_lista(Lista, Ordinata) :-
    ordina_lista(Lista, [], Ordinata).

/* Predicato che ordina la lista ricorsivamente. */
ordina_lista([], Acc, Acc).
ordina_lista([X | Xs], Acc, Ordinata) :-
    inserisci_decrescente(X, Acc, NuovoAcc),
    ordina_lista(Xs, NuovoAcc, Ordinata).

/* Predicato che inserisce un elemento in una lista mantenendo
   l'ordine decrescente. */
inserisci_decrescente(Punteggio1-Titolo1, [], [Punteggio1-Titolo1]).
inserisci_decrescente(Punteggio1-Titolo1, [Punteggio2-Titolo2 | Rest], [Punteggio1-Titolo1, Punteggio2-Titolo2 | Rest]) :-
    Punteggio1 >= Punteggio2.
inserisci_decrescente(Punteggio1-Titolo1, [Punteggio2-Titolo2 | Rest], [Punteggio2-Titolo2 | NewRest]) :-
    Punteggio1 < Punteggio2,
    inserisci_decrescente(Punteggio1-Titolo1, Rest, NewRest).   
\end{lstlisting}

\newpage
\section{Testing}
\subsection{Testing del programma in Haskell}
\begin{center}
    \textbf{Test 1}
    \par
    \vspace{0.5cm}
    \includegraphics[width=0.5\textwidth]{htest1}
\end{center}
\begin{center}
    \textbf{Test 2}
    \par
    \vspace{0.5cm}
    \includegraphics[width=0.5\textwidth]{htest2}
\end{center}
\begin{center}
    \textbf{Test 3}
    \par
    \vspace{0.5cm}
    \includegraphics[width=0.5\textwidth]{htest3}
\end{center}

\newpage
\begin{center}
    \textbf{Test 4}
    \par
    \vspace{0.5cm}
    \includegraphics[width=0.5\textwidth]{htest4}
\end{center}
\begin{center}
    \textbf{Test 5}
    \par
    \vspace{0.5cm}
    \includegraphics[width=0.5\textwidth]{htest5}
\end{center}

\newpage
\begin{center}
    \textbf{Test 6}
    \par
    \vspace{0.5cm}
    \includegraphics[width=0.5\textwidth]{htest6}
\end{center}
\begin{center}
    \textbf{Test 7}
    \par
    \vspace{0.5cm}
    \includegraphics[width=0.5\textwidth]{htest7}
\end{center}
\begin{center}
    \textbf{Test 8}
    \par
    \vspace{0.5cm}
    \includegraphics[width=0.5\textwidth]{htest8}
\end{center}

\newpage
\begin{center}
    \textbf{Test 9}
    \par
    \vspace{0.5cm}
    \includegraphics[width=0.5\textwidth]{htest9}
\end{center}
\begin{center}
    \textbf{Test 10}
    \par
    \vspace{0.5cm}
    \includegraphics[width=0.5\textwidth]{htest10}
\end{center}
\vspace{1cm}

\newpage
\subsection{Testing del programma in Prolog}
\begin{center}
    \textbf{Test 1}
    \par
    \vspace{0.5cm}
    \includegraphics[width=0.5\textwidth]{ptest1}
\end{center}
\begin{center}
    \textbf{Test 2}
    \par
    \vspace{0.5cm}
    \includegraphics[width=0.5\textwidth]{ptest2}
\end{center}
\begin{center}
    \textbf{Test 3}
    \par
    \vspace{0.5cm}
    \includegraphics[width=0.5\textwidth]{ptest3}
\end{center}

\newpage
\begin{center}
    \textbf{Test 4}
    \par
    \vspace{0.5cm}
    \includegraphics[width=0.5\textwidth]{ptest4}
\end{center}
\begin{center}
    \textbf{Test 5}
    \par
    \vspace{0.5cm}
    \includegraphics[width=0.5\textwidth]{ptest5}
\end{center}
\begin{center}
    \textbf{Test 6}
    \par
    \vspace{0.5cm}
    \includegraphics[width=0.5\textwidth]{ptest6}
\end{center}
\begin{center}
    \textbf{Test 7}
    \par
    \vspace{0.5cm}
    \includegraphics[width=0.5\textwidth]{ptest7}
\end{center}

\newpage
\begin{center}
    \textbf{Test 8}
    \par
    \vspace{0.5cm}
    \includegraphics[width=0.5\textwidth]{ptest8}
\end{center}
\begin{center}
    \textbf{Test 9}
    \par
    \vspace{0.5cm}
    \includegraphics[width=0.5\textwidth]{ptest9}
\end{center}
\begin{center}
    \textbf{Test 10}
    \par
    \vspace{0.5cm}
    \includegraphics[width=0.5\textwidth]{ptest10}
\end{center}
\vspace{1cm}

\end{document}